\documentclass{article}
\usepackage[utf8]{inputenc}
\usepackage[american]{babel}

\usepackage[backend=biber,refsection=section,sorting=ydnt,style=authoryear]{biblatex}
\usepackage{biblatex-tweaks}

\usepackage{hyperref}
\addbibresource{/home/makmiller/Documents/mydocs/websites/website-main/latex/tex-bib/my-pubs.bib}

\usepackage{graphicx}

\usepackage{lineno}
\usepackage{setspace}

\author{ }

\date{}

\title{Research}

\begin{document}

% \maketitle
\onehalfspacing{}

% \tableofcontents

%% ===============
%% custom commands
%% ===============

\renewcommand{\url}[1]{\href{#1}{\underline{PDF}}}

%% ====
%% Text
%% ====

\section{The Ecology of Cooperation \& Conflict}

\nocite{pedrosoed:_impac}

\printbibliography[heading=none]

\section{The Evolution of Biological Individuals}

One striking feature about life is that, instead of being distributed
homogeneously on Earth's surface, it is clustered into biological
individuals, such as ants, dogs, and beehives. But why would
free-living individuals (e.g., single cells) forgo their independent
existence and merge into higher-level individuals (e.g., multicellular
organisms)? The goal of this project is to address this issue by using
biofilms and other microbial communities as case studies.

\nocite{pedrosong:_formin_lineag_stick_toget,pedroso2016inheritance,
  ereshefskyth:_rethin_evolut_indiv,
  ereshefskyth:_what_biofil_can_teach_us_about_indiv,
  ereshefskyss:_biolog}

\printbibliography[heading=none]


\section{The Role of Common Ancestry in Biological Classification}

The classification of organisms into species and higher taxa is often
genealogical. For instance, \emph{Homo sapiens} belongs to the order
Primates because humans descended from earlier primates. Once we take
a closer look at how common ancestry fixes the membership conditions
of biological taxa, however, a series of problems opens up. `What is a
common ancestor?' `Are microbial species best viewed as lineage
segments given that microbes can exchange genes horizontally?' The
goal of this project is to investigate these and other problems taking
into account the role of common ancestry in biological classification.

\nocite{pedroso12:_essen_histor_biolog_taxa, pedroso2013origin}

\printbibliography[heading=none]

\end{document}

%%% Local Variables:
%%% TeX-master: t
%%% End:
